\chapter{Tolerância a Falhas}
\label{cha:tolerancia_a_falhas}

\section{Versão 1: C++}
\label{sec:versao1_cpp}
A variante de \textit{c++}, possui vários mecanismos de tolerância a falhas, tanto na leitura do ficheiro dos dados
de input, como no tratamento e processamento desses mesmos dados.\par
Em primeiro lugar todos os inputs fora dos limites da função fornecida para o cálculo da glucose são automaticamente
descartados. Posteriormente é verificado o estado de cada sensor (operacional ou inoperacional), para tal existem
vários parâmetros a ter em conta nomeadamente se o sensor está preso \textit{stuck}, este estado é determinado 
comparando a leitura atual com a leitura anterior e a leitura anterior é comparada com a imediatamente antes, se a diferença entre
estas duas comparações for inferior a 0.01, então o sensor é considerado \textit{stuck}. \par Um outro parâmetro a ter
em conta é se a sucessão dos valores lidos pelos sensores são díspares, ou seja se existe uma grande variação entre
eles, para tal a cada leitura, é verificado se a diferença entre a leitura atual e a anterior é superior ao limite
estabelecido de duas unidades. Foi estabelecido este limite porque considera-se que é fisicamente impossível haver
tal variação durante 1 minuto. Quando existe uma falha de comunicação, ou seja recebe um input da forma '--', o valor
do sensor nesse minuto é automaticamente descartado.\par Depois de fazer todas estas validações, o objetivo é juntar as
leituras dos dois sensores numa só leitura. Se os dois sensores forem considerados operacionais é usada a média das
leituras, se apenas existir um sensor operacional é usada a leitura deste sensor, por fim se nenhum sensor estiver
operacional a leitura deste minuto é descartada.\par Posteriormente para calcular de 3 em 3 minutos a dose de insulina
a ser injetada, se o valor da glucose no sangue em mmol/litro não possuir um valor válido no minuto coicidente com o
terceiro minuto, são verificados os valores calculados para os dois minutos anteriores, se o valor da glucose
imediatamente anterior for válido é usado esse valor para calcular a dose, senão é usado o anterior a esse. Quando os
dois valores anteriores forem válidos, é calculada a média aritmética entre eles, caso nenhum destes seja válido então
o programa retorna \textit{FAIL}. No extremo oposto se o valor da glucose o sangue possuir um valor válido no minuto
coicidente com o terceiro minuto é utilizado este valor para calcular a dose a injetar no paciente.

\section{Versão 2: Java}
\label{sec:versao2_java}

Esta versão do programa começa por ler o ficheiro onde se encontram os dados obtidos pelos sensores. Imediatamente após
esta operação, o programa faz um processamento das leituras obtidas pelos sensores rejeitando imediatamente todas as que
não se encontrem em confirmidade com o domínio da função de calcúlo da glucose, ou seja valores não pertencentes ao
domínio de [1,5].\par Uma vez terminado este processamento os dados recolhidos pelos sensores serão todos analisados
numa tentativa de validar os resultados recolhidos, tentando que o programa seja o mais tolerante possível a eventuais
falhas nos sensores.\par Para que tal seja possível, diversas verificações são efectuadas. Em primeiro lugar se para um
determinado minuto ambos os sensores forem incapazes de obter uma leitura, seja por uma falha de comuniação, ou não,
para esse minuto os valores são automaticamente discartados.\par Para as proximas verificações é necessário explicar que
os sensores podem-se encontrar em vários estados. Os sensores podem estar presos sempre no mesmo valor, podem estar a
receber valores aleatórios.\par Seguidamente a esta verificação, é verificado se algum dos sensores está com problemas
de comunicação, nao produzindo desta forma leituras, se isto se verificar para algum sensor é verificado se o outro
sensor está a produzir uma leitura válida, se sim é considerado o valor deste sensor, caso o segundo sensor esteja preso
ou a receber valores aleatórios ambos os valores medidos são rejeitados.\par Para o caso em que ambos os sensores se
encontrem a receber leituras, são verificados alguns casos especiais, para ambos os sensores, se algum se encontra preso
ou a receber leituras aleatórias. Se tal se verificar para ambos os sensores é utilizado ambas as leituras são
automaticamente discartadas, se por outro lado tal se verificar apenas para um dos sensores, é utilizada a leitura do
outro sensor.\par De seguida, se ambos os sensores se encontrarem a produzir valores aparentemente válidos, ou seja que
não se enquadrem em nenhum dos casos especiais já mencionados, são ainda assim realizadas algumas validações antes de
aceitar algum valor. Para este caso, se os valores recebidos pelos sensores forem muitos distantes (diferença de pelo
menos 40\%) um do outro, é analisado qual o mais próximo da média das ultimas medições e é considerado esse valor, caso
contario, ou seja no caso em que ambos os valores recebidos são validos é considerado o valor médio das duas medições.\par
Posto isto termina a fase de validação dos valores obtidos pelos sensores. Assim para os valores obtidos de minuto a
minuto é calculado o valor da glucose. Completo este passo apenas resta de 3 em 3 minutos calcular a dose de insulina a
injetar no paciente. Para tal, são também realizadas algumas verificações. Apesar de só ser injentada insulina de 3 em
3 minutos o processamento da quantidade de insulina a injetar é realizado em blocos de 3 minutos. Ou seja, se para o
minuto em que deveria ser injetada insulina existe uma leitura de glucose é calculada a dose a injetar com base nesse
mesmo valor. Contudo para o caso em que por falha dos sensores não existe um valor de glucose para esse minuto, o
programa calcula o valor da dose a injetar para o primeiro e segundo valores do bloco de 3 minutos e calcula a média
entre eles obtendo assim a dose a injetar, caso tal não seja possível por nalgum dos minutos não haver um valor válido
de glucose, o programa tenta calcular a dose tendo por base o valor da glucose do segundo minuto e caso tal não seja
possível usando o do primeiro minuto. Contudo pode dar-se o caso de nenhuma destas hipóteses ser possível, nesse caso o
programa retorna \textit{FAIL}.

\section{Versão 3: Python}
\label{sec:versao3_python}
A variante em \textit{Python} tem vários mecanismos de tolerância a falhas, tanto na leitura do ficheiro com os dados
de glucose no sangue como no tratamento desses dados e na escrita do ficheiro com as doses a administrar ao paciente.\par
Na leitura do ficheiro o programa falha se o mesmo não existir ou se o ficheiro não for passado ao programa.\par
Depois de obter os dados do ficheiro o programa vai tratar desses mesmos dados seguindo um conjunto de restrições.
A primeira restrição aplicada ao conjunto de dados tem a ver com o domínio, só são válidos valores entre 1 e 5, exclusive.
A segunda restrição verifica se os valores lidos num dado minuto têm uma diferença superior a 30\%, se tiverem o programa
incrementa um contador e se o valor desse contador for superior à tolerância de sincronização definida, neste caso 3,
ambas as entradas dos sensores são descartadas se a diferença for inferior aos 30\% esse contador é decrementado, esta
restrição garante que os valores dados por ambos os sensores não são muito diferentes.
A terceira restrição verifica se o sensor ficou preso num certo valor, para isso ele verifica se o valor que está a ler
de um certo sensor é igual às duas últimas leituras válidas desse mesmo sensor, se sim esse valor é descartado.
A última restrição aplicada aos dados de entrada é uma tentativa de minimizar valores aleatórios, esta restrição
consiste em descartar valores que variem em mais de 30\% da última leitura válida.\par
Depois dos valores de entrada serem tratados o programa combina os valores dos dois sensores num único valor, esta
combinação é feita da seguinte forma: caso as duas leituras sejam inválidas, ou seja, o sensor não leu ou a leitura não
era válida segundo as restrições aplicadas, o estado retornado nesse minuto é de \textit{FAIL}; caso uma das leituras seja
inválida o estado retornado nesse minuto é a leitura válida; caso ambas sejam válidas é retornada a média entre os dois
valores.\par
Para o cálculo da dosagem a administrar ao cliente o programa apenas utiliza valores de três em três minutos, se o valor
do terceiro minuto não for válido é utilizado o valor do segundo minuto e se esse não for válido é utilizado o valor do
primeiro minuto, caso nenhum deles seja válido é retornado \textit{FAIL} nessa dosagem. Se houver um valor válido nesses
três minutos e se esse valor for maior que 6 é calculada a tendência tendo em conta os últimos valores, até um máximo
de 5. Se a têndencia for de descida e maior que 0.4 por minuto é calculada a dose usando as fórmulas fornecidas no
enunciado do trabalho. Ainda é feita uma verificação com as fórmulas da glucose e das suas derivadas com um valor
conhecido para verificar se a memória está corrompida.\par
Calculadas as doses a administrar estas são escritas no ficheiro que foi passado ao programa.\par
Foi usado uma variação máxima de 30\% entre os valores nas restrições, para não permitir variações fora do que é
fisicamente possível. Para o cálculo da tendência foi usado um histórico de no máximo 5 leituras, pois os níveis de
glucose no sangue tendem a ter uma variação uniforme e lenta e este histórico permite calcular uma tendência muito
próxima da realidade.

\section{Votador}
\label{sec:votador}
O votador é um mecanismo simples, desenvolvido em \textit{Python}, de escolha das doses a administrar ao paciente, tendo
em conta o resultado dos cálculos das três versões acima descritas.\par
A escolha da dose é feita tendo em conta os seguintes passos: se as todas as doses forem inválidas, o votador retorna
\textit{FAIL}; se alguma das doses for válida, ele retorna essa dose; se houver mais do que uma dose válida, o votador
verifica se a diferença entre as doses é superior a 2, caso nenhuma das doses cumpra este requisito ele retorna
\textit{FAIL}, caso contrário ele retorna a média das duas ou três doses que cumprem esta restrição. A diferença
máxima entre as doses de cada versão é de dois, porque permite garantir que as doses não se afastem muito do valor
correto, o que é reforçado pelo uso da média entre os valores válidos.
